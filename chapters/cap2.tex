\chapter{Stage}
\label{cap:Stage}

\section{Visione aziendale}
\subsection{Offerte aziendali}
(In questa sezione inquadro la visione generale dell’azienda
riguardo la formazione e l’innovazione, descrivendo come i programmi di stage si
inseriscono in questa strategia sia per lo scouting tecnologico che per l’inserimento di nuove
risorse)
\subsection{Stage in azineda}
(In questa sezione descrivo l’obiettivo aziendale strategico negli
stage, ovvero formare futuri professionisti e avere la possibilità di valutare nuove tecnologie
oltre a fornire un punto di vista pratico nel mondo del lavoro.)
\subsection{Ruolo del tutot aziendale}
(In questa sezione descrivo il ruolo del tutor aziendale,
come avveniva la comunicazione con questa figura e in che modo ha contribuito alla mia
formazione, quindi in generale al suo essere guida, supporto e supervisore)
\section{Motiazione dello stage}
(in questa sezione evidenzio come lo stage sia stato parte di
una strategia di scouting tecnologico per valutare la fattibilità e l'introduzione del framework
in nuovi prodotti)
\section{Progetto proposto}
(in questa sezione descrivo il prodotto da realizzare proposto
dall'azienda, parlando delle sfide preventivate ad inizio progetto e cosa dovevo realizzare)
\section{Obiettivi e vincoli}
(in questa sezione descrivo gli obiettivi specifici del mio stage,
specificando i rischi e le contromisure adottate per il conseguimento degli obiettivi)
\section{Scelta dello stage}
\subsection{Le motivazioni della scelta}
(in questa sezione discuto sulle ragioni che mi hanno
portato a preferire questa offerta di stage rispetto ad altre, in particolar modo sull'opportunità
di lavorare a un progetto ad alta autonomia ed innovazione.)
\subsection{Obiettivi personali prefissati}
(in questa sezione discuto dei risultati che mi ero
riproposto di raggiungere al termine dello stage)



