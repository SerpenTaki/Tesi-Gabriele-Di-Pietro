\chapter{Stage}
\label{cap:Stage}

\section{Visione aziendale}
\subsection{Offerte aziendali}
Per \textit{SyncLab}, i programmi di \textit{stage} rappresentano uno strumento strategico per favorire l’innovazione interna e sostenere l’evoluzione continua delle competenze aziendali. L’azienda considera infatti l’inserimento di stagisti come un’opportunità per acquisire nuove conoscenze e approfondire aree emergenti dell’\textit{information technology}, integrando tali contributi nei progetti esistenti o cogliendo l'opportunità di esplorare tecnologie non ancora adottate.
Lo \textit{stage} diventa così un mezzo per cercare soluzioni e idee che, nell'impegno continuo sulle attività operative quotidiane, difficilmente verrebbero affrontate. Al tempo stesso, offre allo stagista un contesto professionale nel quale sviluppare nuove competenze e specializzarsi in ambiti di interesse. Questo modello è reso possibile da una rete consolidata di collaborazioni con università italiane ed estere, che garantisce un continuo scambio di conoscenze e talenti.

\begin{figure}[!ht]
    \centering
    \includegraphics[width=\textwidth]{../assets/collab.png} 
    \caption{Collaborazioni dell'azienda - Fonte: synclab.it}
\end{figure}

\newpage

I programmi di \textit{stage} si articolano principalmente in tre aree:
\begin{itemize}
    \item \textbf{Integrazione}: gli stagisti contribuiscono al perfezionamento di \textit{software} già in uso, intervenendo su funzionalità specifiche. Tale attività consente di ampliare la comprensione delle potenzialità delle soluzioni aziendali e di migliorarne progressivamente le \textit{performance}.
    \item \textbf{Analisi} e \textbf{ottimizzazione}: questa area prevede una valutazione approfondita delle soluzioni \textit{software} esistenti, con l’obiettivo di individuare inefficienze, proporre miglioramenti e ottimizzare i processi. L’approccio critico e analitico adottato permette di valorizzare al massimo il patrimonio tecnologico dell’azienda.
    \item \textbf{Innovazione}: in questo ambito gli stagisti svolgono analisi teoriche e sperimentazioni su tecnologie emergenti, con \textit{focus} specifici allineati agli obiettivi strategici aziendali. Questa attività valuta il potenziale innovativo delle nuove tendenze e di identificare possibili applicazioni operative.
\end{itemize}

\begin{figure}[!ht]
    \centering
    \includegraphics[width=0.8\textwidth]{../assets/stage.png} 
    \caption{Contributo dei programmi di \textit{stage} in SyncLab.}
    \label{fig:ciclo_stage}
\end{figure}

\newpage 
\subsection{Stage in azienda}
Gli \textit{stage} presso \textit{SyncLab} rappresentano un investimento strategico bidirezionale, che riflette la duplice identità aziendale, orientata alla crescita interna e all'avanguardia tecnologica. I programmi sono concepiti come un ciclo continuo di apprendimento, dove l'azienda trae valore dalle competenze degli stagisti quanto questi ultimi traggono esperienza dall'organizzazione.
Questo approccio si manifesta attraverso due obiettivi fondamentali:
\begin{enumerate}
    \item \textbf{Formazione e inserimento di risorse qualificate:} Il programma di \textit{stage} funge da canale primario per identificare e valutare talenti. L'azienda offre un'immersione pratica, consentendo loro di applicare le conoscenze accademiche a problemi reali. La società valuta in modo approfondito il potenziale dei candidati, e gli stagisti che mostrano valore e impegno ricevono offerte di assunzione come membri effettivi del \textit{team}. Questo garantisce all'azienda una \textit{pipeline} costante di nuove risorse già integrate nel proprio ecosistema.
    \item \textbf{Acquisizione di nuove competenze:} Il programma di \textit{stage} funge da banco di prova per l'innovazione continua, riducendo così i rischi associati all'adozione di nuove tecnologie. Attraverso questo meccanismo l'azienda rimane all'avanguardia e adatta le competenze consolidate a contesti tecnologici inediti.
\end{enumerate}

\subsection{Ruolo del tutor aziendale}
Durante il percorso ad ogni stagista è assegnato a un \textit{tutor}. Il \textit{tutor} è una figura professionale che opera nello stesso ambito in cui lo stagista svolge le proprie attività costituendo di fatto un punto di riferimento. Il suo ruolo ha tre funzioni principali:
\begin{itemize}
    \item \textbf{Guida}: il \textit{tutor} propone linee guida su come organizzare il lavoro e fornisce allo stagista materiali utili che possono aiutarlo alla comprensione delle tecnologie da utilizzare nel progetto.
    \item \textbf{Supporto}: il \textit{tutor} offre una serie di suggerimenti o soluzioni possibili ai problemi incontrati, questo per scongiurare l'eventualità che il lavoro venga bloccato per lungo tempo e per garantire allo stagista un confronto con una figura preparata stimolando una soluzione concreta.
    \item \textbf{Supervisore}: il \textit{tutor} offre \textit{feedback} allo stagista nella valutazione del lavoro svolto per verificarne la qualità e supportarlo al miglioramento. Questa attività è molto utile in quanto migliorativa nei confronti dello stagista per formarlo verso un percorso professionale.
\end{itemize}
Nel mio caso i dialoghi con il \textit{tutor} assegnato sono stati molto importanti in quanto senza il suo confronto, e i suoi \textit{feedback}, avrei incontrato maggiori ostacoli, ed i conseguenti ritardi, nello sviluppo del progetto. Anche i suggerimenti sull’architettura di sistema, l’organizzazione delle cartelle, e un congruo supporto all’analisi del problema sono state fondamentali per la progettazione dell’applicazione e per la comprensione dei vari componenti da utilizzare.

\newpage

\section{Motivazione dello \textit{stage}}
Lo \textit{stage} si è posto l’obiettivo di valutare la fattibilità e il potenziale impiego di nuove tecnologie nello sviluppo di prodotti futuri. Lo sviluppo nativo per \textbf{Android} e \textbf{iOS} richiede competenze eterogenee e la gestione di \textit{codebase} separate, con conseguenti costi di manutenzione e sviluppo elevati che potrebbero rappresentare un costo elevato. Un'altra dimensione riguarda l’utilizzo di una cifratura \textit{end-to-end} che consente di rendere dei messaggi inseriti nell’applicazione illeggibili ad un intercettatore, implementando quindi un’applicazione mobile capace di generare e gestire dei \textit{wallet} crittografici, comprensivi di coppie private e pubbliche.
Per questo motivo, lo stage ha previsto lo studio e la valutazione del \textit{framework} Flutter e linguaggio Dart, che consentono di sviluppare un applicazione mobile per diversi dispositivi tramite un’unica \textit{codebase}. 

\section{Progetto proposto}
Nel contesto attuale, i dispositivi mobili rappresentano uno strumento di comunicazione ampiamente diffuso e accessibile.
Tuttavia, l’utilizzo di canali comuni espone gli utenti a potenziali rischi di intercettazione e compromissione della \textit{privacy}.
In questo scenario, la protezione delle comunicazioni assume un ruolo centrale e richiede soluzioni tecniche in grado di garantire riservatezza e integrità dei messaggi scambiati.
\newline
Si richiede la progettazione e l’implementazione di un’applicazione in grado si generare e gestire \textit{wallet} crittografici, comprensivi di coppie di chiavi private e pubbliche.
\newline
La realizzazione del progetto ha comportato l'identificazione e la risoluzione delle seguenti sfide progettuali e tecnologiche.
\subsubsection*{Sviluppo multipiattaforma}
 L’applicazione deve essere sviluppata interamente tramite il \textit{framework} Flutter, e deve poter girare su dispositivi diversi con architetture diverse, sia \textit{Android} che \textit{iOS} a partire da un unica base di codice in modo tale che sia mantenibile senza l’impiego di numerose risorse. Questo richiede uno studio approfondito del \textit{framework} e del linguaggio di programmazione Dart, che potrebbe rappresentare un onere importante.
\subsubsection*{Algoritmi di crittografia}
Sviluppare un applicazione che permetta di verificare che gli algoritmi di crittografia vengano impiegati correttamente in modo tale che i messaggi inseriti nell’applicazione vengano letti solo da interessati e non da esterni. Questo richiede uno stratagemma oppure lo sviluppo di qualcosa che consenta allo \textit{stakeholders} di vedere come i messaggi vengano criptati e decriptati.
\subsubsection*{Gestione delle chiavi crittografiche} 
Il recupero della chiave privata usata per decifrare un messaggio rappresenta una grossa sfida in quanto l’applicazione deve poter generare più chiavi crittografiche e sopratutto l’applicazione deve supportare il passaggio dati da un dispositivo ad un altro nuovo in caso di smarrimento. Quindi ci deve essere una base di dati che consenta il recupero di informazioni sul dispositivo, ma alcune di queste informazioni devono essere trasmesse in modo differente. 
\subsubsection*{Integrazione con il lettore NFC}
L'applicazione deve supportare l'utilizzo del lettore NFC presente sui dispositivi mobili in modo da associare i documenti con chip RFID validi alle chiavi salvate sul dispositivo. Quindi bisogna capire che dati si possono prendere dai documenti tramite l'utilizzo di librerie Flutter/Dart e come poterle integrare nell'applicazione.

\section{Obiettivi e vincoli}
\subsection{Obiettivi}


\subsection{Vincoli}



\section{Scelta dello \textit{stage}}
\subsection{Motivazioni della scelta}
Ho conosciuto \textit{Synclab} nel mese di aprile attraverso una comunicazione telefonica, inizialmente ho rifiutato l'opportunità proposta poiché intendevo valutare altre possibilità. In seguito, durante il mese di giugno, ho deciso di ricontattare l’azienda, che mi ha presentato due proposte di progetto particolarmente interessanti: la prima riguardava lo sviluppo di un’applicazione con backend basato su Odoo ERP, mentre la seconda consisteva nella realizzazione di un’applicazione mobile sviluppata tramite il framework Flutter, dedicata alla generazione di wallet contenenti chiavi pubbliche e private.

Parallelamente, durante il periodo estivo avevo ricevuto ulteriori offerte di stage da altre aziende e stavo valutando alcune alternative per avviare un percorso lavorativo tempestivamente, nella consapevolezza che l’inizio effettivo dello stage sarebbe comunque avvenuto a settembre per la pausa estiva. 

Alla fine ho scelto di intraprendere il mio percorso in \textit{Synclab} per i seguenti motivi:
\begin{itemize}
    \item \textbf{Possibilità di lavorare da remoto}, condizione per me fondamentale poiché, dovendomi spostare frequentemente e non disponendo di un mezzo di trasporto personale, avrei avuto difficoltà a raggiungere quotidianamente la sede aziendale.
    \item \textbf{Elevato livello di autonomia offerto dal progetto}, trattandosi di uno \textit{scouting} tecnologico finalizzato alla valutazione dei punti di forza e delle criticità del framework Flutter. Ciò mi avrebbe permesso di affrontare un lavoro altamente esplorativo, con ampi margini decisionali.
\end{itemize}

\subsection{Obiettivi personali prefissati}
Nell'attuale contesto tecnologico, l’uso dei dispositivi mobili è divenuto parte integrante della vita quotidiana: la maggior parte delle persone possiede almeno uno smartphone, attraverso il quale comunica e accede a numerosi servizi. Tuttavia, molte applicazioni di messaggistica sono oggi controllate dalle \textit{big tech} e sia l’hardware sia il \textit{software} risultano sempre più gestiti da grandi compagnie e sempre meno direttamente dagli utenti.

L’introduzione della proposta di legge europea relativa al controllo delle chat e alla possibilità di analizzare i dati delle conversazioni ha ulteriormente alimentato in me domande sulla reale tutela dei dati personali e sensibili. Le chiavi private degli utenti, che idealmente dovrebbero rimanere sotto il loro esclusivo controllo, sono spesso memorizzate su server esterni per ragioni di praticità.

Da qui è nata la domanda che ha guidato parte della mia motivazione: esiste un modo per conservare le chiavi private in spazi realmente sicuri all'interno del dispositivo, garantendo una protezione efficace e impedendo l’accesso da parte di terzi?

Nello sviluppo dell’applicazione mi ero quindi posto i seguenti obiettivi:
\begin{itemize}
    \item Progettare un’applicazione capace di gestire i dati privati degli utenti in modo distinto rispetto a quelli pubblici, incrementandone la sicurezza complessiva;
    \item Apprendere lo sviluppo di applicazioni per dispositivi mobili attraverso l’utilizzo del framework Flutter;
    \item Acquisire i principi di crittografia necessari a consentire una comunicazione sicura tra due dispositivi mobili.
\end{itemize}

In conclusione, lo stage presso \textit{Synclab} costituiva per me un’importante occasione per approfondire i concetti di sicurezza informatica affrontati durante il percorso universitario e applicarli in un contesto reale, concreto e di grande rilevanza come il settore delle applicazioni mobili.