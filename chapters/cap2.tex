\chapter{Stage}
\label{cap:Stage}

\section{Visione aziendale}
\subsection{Offerte aziendali}
(In questa sezione inquadro la visione generale dell’azienda riguardo la formazione e l’innovazione, descrivendo come i programmi di stage si inseriscono in questa strategia sia per lo scouting tecnologico che per l’inserimento di nuove risorse)
Ogni anno l'azienda \textit{Synclab} partecipa con l'università degli studi di Padova 



\subsection{Stage in azineda}
(In questa sezione descrivo l’obiettivo aziendale strategico negli stage, ovvero formare futuri professionisti e avere la possibilità di valutare nuove tecnologie oltre a fornire un punto di vista pratico nel mondo del lavoro.)



\subsection{Ruolo del tutot aziendale}
(In questa sezione descrivo il ruolo del tutor aziendale, come avveniva la comunicazione con questa figura e in che modo ha contribuito alla mia formazione, quindi in generale al suo essere guida, supporto e supervisore)


\section{Motiazione dello stage}
(in questa sezione evidenzio come lo stage sia stato parte di una strategia di scouting tecnologico per valutare la fattibilità e l'introduzione del framework in nuovi prodotti)



\section{Progetto proposto}
(in questa sezione descrivo il prodotto da realizzare proposto dall'azienda, parlando delle sfide preventivate ad inizio progetto e cosa dovevo realizzare)



\section{Obiettivi e vincoli}
(in questa sezione descrivo gli obiettivi specifici del mio stage, specificando i rischi e le contromisure adottate per il conseguimento degli obiettivi)


\section{Scelta dello stage}
\subsection{Le motivazioni della scelta}
(in questa sezione discuto sulle ragioni che mi hanno portato a preferire questa offerta di stage rispetto ad altre, in particolar modo sull'opportunità di lavorare a un progetto ad alta autonomia ed innovazione.)
\\
\\
Ho conosciuto \textit{Synclab} ad Aprile tramite via telefonica che ho inizialmente rifiutato in quanto avevo intenzione di proseguire su altre strade.
Tuttavia ho provedduto a ricontattarla durante il periodo estivo di giugno.
L'azienda mi ha quindi proposto di lavorare a un paio di progetti interessanti:
il primo trattava di un applicazione con backend Odoo ERP, mentre la seconda trattava di un applicazione mobile realizzata tramite il framework flutter che permetteva di generare wallet con chiavi pubbliche e private.
Dopo aver valutato alcune offerte di lavoro ricevute anche da altre aziende durante il periodo estivo ed aver cercato qualche soluzione alternativa in modo da mettermi subito al lavoro, in quanto non avrei iniziato prima di settembre.
Ho deciso di iniziare il mio percorso di stage in \textit{Synclab} per diversi motivi:
\begin{itemize}
    \item Il primo era quello di lavorare da remoto in quanto per diversi motivi sono costretto a muovermi lontano da Padova e non dispongo di un automobile personale o altri mezzi.
    \item Il secondo perchè avrei avuto enorme autonomia nel progetto in quanto si trattava di uno \textit{scouting} tecnologico e quindi all'azienda serviva verificare i punti di forza e di debolezza di Flutter.
\end{itemize}

\subsection{Obiettivi personali prefissati}
(in questa sezione discuto dei risultati che mi ero riproposto di raggiungere al termine dello stage)
\\
\\
Nel mondo odierno lo sviluppo della tecnologia e della società ci ha sempre portato di più ad utilizzo
costante dei dispositivi mobili, ad oggi ogni persona deve almeno possedere un numero di telefono collegarsi
e comunicare con altre persone. Tuttavia molto spesso la maggiorparte delle applicazioni di messaggistica che permettono
scambi di messaggi con le altre persone sono sempre di più in mano alle \textit{big tech}, e sia i dispositivi che 
i \textit{software} installati su essi sono sempre meno posseduti dagli utenti e sempre più in mano alle grandi compagnie.
Con l'uscita della proposta di legge sul controllo delle chat da parte dell'unione europea e la richiesta di analizzare i dati 
delle conversazioni da parte degli utenti mi sono sempre chiesto di più quanto i dati personali e privati delle
persone possano essere al sicuro. Le chiavi private degli utenti dovrebbero essere messe al sicuro e gestite dagli utenti stessi
invece per comodità sono salvate su dei server.
Quindi esiste un modo per salvare le chiavi privati in spazi sicuri o tag in modo tale da mantenere i
dati privati al sicuro e indecifrabili all'esterno.
Quindi nello sviluppo dell'applicazione per l'azienda mi ero riproposto di:
\begin{itemize}
    \item Creare un applicazione che gestisse i dati privati degli utenti in modo diverso da quello pubblico
    \item Imparare a sviluppare applicazioni per dispositivi mobili
    \item Imparare i principi di crittografia per permettere la comunicazione tra 2 dispositivi mobili
\end{itemize}