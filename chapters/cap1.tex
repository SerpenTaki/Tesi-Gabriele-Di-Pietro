\chapter{Contesto aziendale}
\label{cap:Contesto aziendale}

Il presente capitolo introduce l'azienda \textbf{SyncLab S.r.L}, presso cui ho svolto l'attività di \textit{stage},
fornendo la base necessaria alla comprensione del progetto di tesi. Verranno fornite una descrizione degli
ambiti in cui l'azienda opera, gli strumenti e i processi di \textit{Way of working} di cui ho avuto esperienza
diretta, e la sua propensione all'innovazione, elemento cruciale in relazione allo sviluppo di un'applicazione
innovativa come quella descritta nei futuri capitoli.


\section{Presentazione azienda}
L'azienda ospitante, \textit{SyncLab S.r.L} è stata fondata nel 2002 a Napoli affermandosi fin da subito come una realtà innovativa
attenta ai paradigmi della trasformazione digitale. Nel corso degli anni, ha intrapreso un significatico processo di espansione
a livello nazionale, che l'ha portata ad aprire sei sedi operative distribuite lungo il territorio italiano. 

\begin{figure}[!ht] 
    \centering 
    \includegraphics[width=0.9\columnwidth]{../assets/sediSynclab.png} 
    \caption{Sedi operative dell'azienda - Fonte: synclab.it}
\end{figure}

Inizialmente nata come \textit{software house}, l'azienda si è dedicata allo sviluppo di soluzioni \textit{software} innovative,
progettate ex novo in base alle opportunità e alle esigenze del mercato. In tale contesto, l'azienda si distingue per la sua
per la sua vocazione all'innovazione tecnologica e alla trasformazione digitale, realizzando prodotti e fornendo servizi in diversi
settori strategici, tra cui quello sanitario, energetico, industriale, finanziario e logistico.
\newline
\newline
Nel tempo, \textit{SyncLab S.r.L} ha ampliato il proprio raggio d'azione, assumendo un ruolo rilevante come \textit{system integrator}.
In questa veste l'azienda si occupa dell'ottimizzazione, integrazione e manutenzione di soluzioni \textit{software} già esistenti
per conto di clienti esterni, offrendo supporto tecnologico e consulenziale volto a a favorire l'adozione delle più recenti innovazioni digitali.
Questa evoluzione ha permesso all'azienda di diventare uno dei principali \textit{system integrator} del panorama italiano nel settore
dell'\textit{Information and Communication Technology (ICT)}.
\newline
Questa sua duplice identità coniuga la creatività e la proattività di una \textit{software house} con l'approccio orientato all'efficienza e al cliente tipico
di un \textit{system integrator}, rappresenta uno dei principali punti di forza dell'azienda offrendo soluzioni complete e personalizzate.
\newline
\textit{SyncLab S.r.L} promuove attivamente la collaborazione interna, incoraggiando l'interazione non solo tra i membri
della stessa sede, ma anche tra i colleghi di sedi diverse. Favorendo un costante scambio di conoscenze e competenze,
creando un ambiente dinamico in cui la crescita professionale di ogni individuo è il risultato del lavoro di squadra.

\section{Rami aziendali e progetti}
\textit{SyncLab S.r.L} collabora con un ampio numero di clienti appartenenti a molteplici settori industriali e tecnologici.
Nel corso degli anni l'azienda ha consolidato una presenza trasversale in diversi ambiti, tra cui \textit{web, mobile, privacy, sanitario, blockchain e trasporti}, sviluppando soluzioni capaci di rispondere a esigenze eterogenee.
\begin{figure}[!ht] 
    \centering 
    \includegraphics[width=0.9\columnwidth]{../assets/prodotti.png} 
    \caption{Alcuni prodotti offerti dall'azienda - Fonte: synclab.it}
\end{figure}
\newline
Alcuni dei \textit{software} che l'azienda ha prodotto sono:
\begin{itemize}
    \item \textbf{Sobereye} (ambito \textit{web}): un'applicazione innovativa progettata a monitorare il rischio di deterioramento neuro-cognitivo dovuto a stanchezza, malori, alcol o stupefacenti all'interno dell'ambiente lavorativo.
    \item \textbf{SynClinic} (ambito sanitario): un sistema integrato che supporta la gestione completa dei processi clinici e amministrativi di ospedali, cliniche e case di cura, consentendo di organizzare e monitorare tutte le fasi del percorso di cura del paziente.
    \item \textbf{DPS 4.0} (ambito \textit{web} e \textit{privacy}): una piattaforma \textit{web} che supporta i titolari, responsabili, \textit{data protection officer} (DPO) nelle attività di conformità del Regolamento Generale Protezione Dati (GPDR) nel rispetto del principio di \textit{accountability}.
    \item \textbf{Fast Ride} (ambito trasporti): una soluzione per la gestione di servizi di trasporto pubblico a chiamata in contesti urbani ed extraurbani, pensata per integrare le linee tradizionali con un sistema flessibile, dinamico ed ecocompatibile.
\end{itemize}

\section{Way of working}
In questa sezione tratterò di alcune tecnologie di cui ho avuto esperienza diretta per lo sviluppo dell'applicazione.
\subsection{Tecnologie interne}
L'azienda utilizza una vasta gamma di tecnologie che includono linguaggi di programmazione e \textit{framework} all'avanguardia. 
Per la mia esperienza e lo sviluppo di un applicazione mobile ho utilizzato:
\begin{figure}[!ht] 
    \centering 
    \includegraphics[width=0.6\columnwidth]{../assets/sviluppoMobile.png} 
    \caption{Tecnologie impiegate nella realizzazione dell'applicativo.}
\end{figure}
\begin{itemize}
    \item \textbf{Dart}: un linguaggio di programmazione orientato agli oggetti sviluppato da Google noto per la sua versabilità. Offre numerosi vantaggi come:
        \begin{itemize}
            \item lo sviluppo di un applicativo \textit{cross-platform} che consente di creare applicazioni native per diversi sistemi da un'unica base di codice.
            \item compilazione \textit{Ahead-Of-Time (AOT)} che accelerano lo sviluppo.
            \item supporto alla programmazione asincrona che riduce la gestione dei processi in \textit{background}.
        \end{itemize}
    \item \textbf{Flutter}: un \textit{framework} basato su Dart per lo sviluppo di applicazioni multi-piattaforma, che offre vantaggi come:
        \begin{itemize}
            \item \textit{\textbf{Hot-reload}}: che ci permette di visualizzare immediatamente le modifiche al codice durante lo sviluppo senza dover aspettare una nuova compilazione.
            \item \textbf{Compatibilità con \textit{material}}: ci mette a disposizione un ricco arsenale di \textit{widget} per creare interfacce utente moderne e accattivanti senza dover scrivere molte righe di codice.
        \end{itemize}
    \item \textbf{Firebase}: un \textit{database} sviluppato da Google e ben integrato con Flutter che offre una serie di funzionalità di \textit{backend} pronte all'uso per la propria applicazione.
        \begin{itemize}
            \item \textbf{Authentication}: servizio che offre i servizi di \textit{backend} e librerie pronte all'uso per autenticare gli utenti nell'applicazione in molteplici modi.
            \item \textbf{Firestore}: un \textit{database} \textit{NoSQL} orientato ai documenti, che permette di archiviare i documenti in raccolte, che fungono da contenitori
            per organizzare i dati e facilitare le interrogazioni.
        \end{itemize}
\end{itemize}

Per garantire continuità operativa e un’efficace gestione dei processi collaborativi, è necessario disporre di strumenti in grado di coordinare e sincronizzare le attività del \textit{team}.
Di seguito sono elencati alcuni \textit{software} progettati a questo fine:

\begin{figure}[!ht] 
    \centering 
    \includegraphics[width=0.8\columnwidth]{../assets/sviluppoSupporto.png} 
    \caption{Strumenti di supporto.}
\end{figure}

\begin{itemize}
    \item \textbf{Git}: un sistema di controllo versione distribuito che permette di tracciare in modo efficiente le modifiche ai \textit{file} e di coordinare il lavoro tra più sviluppatori.  
     Questo strumento consente ai membri del gruppo di collaborare in maniera strutturata, gestire eventuali conflitti, apportare modifiche senza rischiare di sovrascrivere il lavoro altrui, creare \textit{branch} dedicati allo sviluppo di nuove funzionalità e recuperare agevolmente versioni precedenti dei \textit{file}.
    \item \textbf{Android Studio}: è un ambiente di sviluppo integrato \textit{(IDE)} gratuito progettato per lo sviluppo di applicazioni \textbf{Android}. Questo \textit{software} mette a disposizione degli sviluppatori diverse funzionalità
     come un sistema di gestione di dispositivi \textit{Android} sia fisici accopiabili tramite cavo o \textit{Wi-Fi}, che virtuali tramite emulatore integrato che ci permette di emulare telefoni \textit{smartphone} o \textit{smartwatch}.
    \item \textbf{Xcode}: è un ambiente di sviluppo integrato \textit{(IDE)} sviluppato e mantenuto da Apple che contiene una \textit{suite} di strumenti utili per lo sviluppo di applicazioni per sistemi proprietari Apple.
    \item \textbf{UMLet}: uno strumento gratuito e \textit{open source} che permette di creare diagrammi UML\footnote{UML: (Unified Modeling Language) è un linguaggio di modellazione e di specifica}. In particolare diagrammi dei casi d'uso, di sequenza e attività e di esportarli in diversi formati.
\end{itemize}
\newpage
\subsection{Smart working e strumenti di comunicazione}
L’azienda adotta un modello di lavoro prevalentemente in remoto: la presenza in sede è richiesta solitamente due volte a settimana, il lunedì e il venerdì, salvo specifiche esigenze. Questi momenti sono dedicati al confronto diretto sul lavoro svolto, ai progressi raggiunti e alle eventuali difficoltà emerse.

Nel mio caso specifico, trattandosi di un progetto di \textit{scouting} tecnologico, gli incontri con il mio responsabile sono stati più frequenti rispetto alla media. Inoltre, ho mantenuto un contatto costante attraverso un server Discord aziendale dedicato, dove aggiornavo regolarmente lo stato di avanzamento, segnalando nuove scoperte, sviluppi dell’applicazione o problemi riscontrati.
\newline
\\
Per garantire una gestione efficace delle attività e tracciare correttamente l’evoluzione del progetto, sono stati utilizzati diversi strumenti digitali:
\begin{itemize}
    \item \textbf{Google Calendar}: un calendario condiviso che permette la creazione e la modifica di eventi, specificandone durata e luogo. È stato utilizzato principalmente per organizzare le presenze in sede, pianificare riunioni e segnalare eventi interni, come ad esempio incontri sindacali.
    \item \textbf{GitHub Projects}: una sezione di GitHub dedicata alla gestione dei \textit{ticket}, utilizzata in alternativa a \textbf{Trello} normalemente utilizzato dall'azienda. Questo strumento consente di creare \textit{roadmap} e schede integrate con le \textit{issue} dei \textit{repository}, facilitando la pianificazione e il monitoraggio delle attività sia a livello individuale che di \textit{team}.
\end{itemize}
Per la comunicazione interna durante il lavoro da remoto:
\begin{itemize}
    \item \textbf{Discord}: utilizzato dai dipendenti per lo scambio di informazioni tramite \textit{chat} testuali e vocali.
     La piattaforma funge anche da spazio per condividere aggiornamenti interni, materiali informativi e risorse formative.
\end{itemize}

\begin{figure}[!ht] 
    \centering 
    \includegraphics[width=0.8\columnwidth]{../assets/sviluppoComunicazione.png} 
    \caption{Strumenti di comunicazione}
\end{figure}

\newpage
\section{Spirito di innovazione aziendale}

L'azienda \textit{SyncLab} si caratterizza per una forte propensione all’innovazione,
 elemento che guida in modo significativo le sue scelte strategiche e operative.
  Questa attitudine si manifesta attraverso un duplice orientamento:
\begin{enumerate}
\item \textbf{Miglioramento delle dinamiche interne}:
 l’azienda investe costantemente nell’ottimizzazione dei processi di gestione dei progetti e nel consolidamento delle relazioni con i clienti.
  Parallelamente, promuove la crescita professionale del \textit{team} tramite aggiornamenti continui su metodologie e tecnologie emergenti, in linea con la propria identità di \textit{system integrator}.
\item \textbf{Proposizione di soluzioni d’avanguardia}:
 L’impegno innovativo non si limita all’organizzazione interna, ma si estende alla progettazione di prodotti e servizi allineati alle prospettive future del mercato. In qualità di \textit{software house}, \textit{SyncLab} sviluppa soluzioni pensate non solo per rispondere alle esigenze attuali dei clienti, ma anche per anticipare \textit{trend} tecnologici, generando valore nel medio-lungo periodo.
\end{enumerate}
In questo contesto orientato alla ricerca dell’avanguardia si colloca la mia esperienza di \textit{stage}. La mentalità aperta al cambiamento e la flessibilità dell’azienda sono state condizioni centrali che hanno reso possibile lo sviluppo di un progetto basato su Dart e Flutter, tecnologie ancora poco diffuse all’interno dell’organizzazione.

\begin{figure}[!ht]
    \centering
    \includegraphics[width=0.75\textwidth]{../assets/progettoStage.png} 
    \caption{Progetto di \textit{stage} come interesezione tra le aree di competenze consolidate di \textit{SyncLab} e l'obiettivo di innovazione tecnologica dell'azienda.}
\end{figure}

\newpage
Il progetto si inserisce coerentemente nella strategia aziendale, contribuendo su due fronti principali:

\begin{itemize}
    \item \textbf{Scouting tecnologico}: L’utilizzo di Flutter per la realizzazione di un’applicazione, caratterizzata da requisiti in ambito di sicurezza (crittografia e NFC \footnote{NFC: Near Field Comunication è una tecnologia di ricetrasmissione che fornisce connettività senza fili bidirezionale a distanza a corto raggio}), si colloca all'interno di una più ampia iniziativa di valutazione dell’adozione di \textit{framework cross-platform} di nuova generazione per i futuri prodotti aziendali.
    \item \textbf{Adattamento di competenze consolidate}: sebbene \textit{SyncLab} possieda già un panorama nella sicurezza e nello sviluppo mobile (come descritto nel Paragrafo 1.2), il progetto ha richiesto la reinterpretazione di alcune conoscenze come la gestione dei \textit{chip} RFID\footnote{RFID: Identificazione a radiofrequenza è una tecnologia di riconoscimento e validazione e/o memorizzazione automatica di informazioni a distanza}/NFC e l’impiego di algoritmi crittografici all’interno di un contesto tecnologico completamente nuovo.
\end{itemize}


L'approccio innovativo di \textit{SyncLab} rappresenta quindi un motore concreto
 che ha reso possibile la realizzazione di una soluzione tecnologicamente avanzata. Il progetto di \textit{stage} ha costituito un vero e proprio banco di prova per l’introduzione di nuove tecnologie, contribuendo alla strategia aziendale di ampliamento e aggiornamento continuo delle competenze interne.