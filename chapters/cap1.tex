\chapter{Contesto aziendale}
\label{cap:Contesto aziendale}
Il presente capitolo introduce l'azienda \textbf{SyncLab S.r.L}, presso cui ho svolto l'attività di stage,
fornendo la base necessaria alla comprensione del progetto di tesi. Verranno fornite una descrizione degli
ambiti in cui l'azienda opera, gli strumenti e i processi di \textit{Way of working} di cui ho avuto esperienza
diretta, e la sua propensione all'innovazione, elemento cruciale in relazione allo sviluppo di un'applicazione
innovativa come quella descritta nei futuri capitoli.


\section{Presentazione azienda}
L'azienda ospitante, \textit{SyncLab S.r.L} è stata fondata nel 2002 a Napoli affermandosi fin da subito come una realtà innovativa
attenta ai paradigmi della trasformazione digitale. Nel corso degli anni, ha intrapreso un significatico processo di espansione
a livello nazionale, che l'ha portata ad aprire sei sedi operative distribuite lungo il territorio italiano. 

\begin{figure}[!ht] 
    \centering 
    \includegraphics[width=0.9\columnwidth]{../assets/sediSynclab.png} 
    \caption{Sedi operative dell'azienda - Fonte: synclab.it}
\end{figure}

Inizialmente nata come \textit{software house}, l'azienda si è dedicata allo sviluppo di soluzioni software innovative,
progettate ex novo in base alle opportunità e alle esigenze del mercato. In tale contesto, l'azienda si distingue per la sua
per la sua vocazione all'innovazione tecnologica e alla trasformazione digitale, realizzando prodotti e fornendo servizi in diversi
settori strategici, tra cui quello sanitario, energetico, industriale, finanziario e logistico.
\newline
\newline
Nel tempo, \textit{SyncLab S.r.L} ha ampliato il proprio raggio d'azione, assumendo un ruolo rilevante come \textit{system integrator}.
In questa veste l'azienda si occupa dell'ottimizzazione, integrazione e manutenzione di soluzioni \textit{software} già esistenti
per conto di clienti esterni, offrendo supporto tecnologico e consulenziale volto a a favorire l'adozione delle più recenti innovazioni digitali.
Questa evoluzione ha permesso all'azienda di diventare uno dei principali \textit{system integrator} del panorama italiano nel settore
dell'\textit{Information and Communication Technology (ICT)}.
\newline
Questa sua duplice identità coniuga la creatività e la proattività di una \textit{software house} con l'approccio orientato all'efficienza e al cliente tipico
di un \textit{system integrator}, rappresenta uno dei principali punti di forza dell'azienda offrendo soluzioni complete e personalizzate.
\newline
\textit{SyncLab S.r.L} promuove attivamente la collaborazione interna, incoraggiando l'interazione non solo tra i membri
della stessa sede, ma anche tra i colleghi di sedi diverse. Favorendo un costante scambio di conoscenze e competenze,
creando un ambiente dinamico in cui la crescita professionale di ogni individuo è il risultato del lavoro di squadra.

\section{Rami aziendali e progetti}
\textit{SyncLab S.r.L} collabora con un ampio numero di clienti appartenenti a molteplici settori industriali e tecnologici.
Nel corso degli anni l'azienda ha consolidato una presenza trasversale in diversi ambiti, tra cui \textit{web, mobile, privacy, sanitario, blockchain e trasporti}, sviluppando soluzioni capaci di rispondere a esigenze eterogenee.
\begin{figure}[!ht] 
    \centering 
    \includegraphics[width=0.9\columnwidth]{../assets/prodotti.png} 
    \caption{Alcuni prodotti offerti dall'azienda - Fonte: synclab.it}
\end{figure}
\newline
Alcuni dei software che l'azienda ha prodotto sono:
\begin{itemize}
    \item \textbf{Sobereye} (ambito web): un'applicazione innovativa progettata a monitorare il rischio di deterioramento neuro-cognitivo dovuto a stanchezza, malori, alcol o stupefacenti all'interno dell'ambiente lavorativo.
    \item \textbf{SynClinic} (ambito sanitario): un sistema integrato che supporta la gestione completa dei processi clinici e amministrativi di ospedali, cliniche e case di cura, consentendo di organizza e monitorare tutte le fasi del percorso di cura del paziente.
    \item \textbf{DPS 4.0} (ambito web e privacy): una piattaforma web che supporta i titolari, responsabili, data protection officer (DPO) nelle attività di conformità del Regolamento Generale Protezione Dati (GPDR) nel rispetto del principio di \textit{accountability}.
    \item \textbf{Fast Ride} (ambito trasporti): una soluzione per la gestione di servizi di trasporto pubblico a chiamata in contesti urbani ed extraurbani, pensata per integrare le linee tradizionali con un sistema flessibile, dinamico ed ecocompatibile.
\end{itemize}

\section{Way of working}
In questa sezione tratterò di alcune tecnologie di cui ho avuto esperienza diretta per lo sviluppo dell'applicazione.
\subsection{Tecnologie interne}
L'azienda utilizza una vasta gamma di tecnologie che includono linguaggi di programmazione e \textit{framework} all'avanguardia. 
Per la mia esperienza e lo sviluppo di un applicazione mobile abbiamo utilizzato:
\begin{itemize}
    \item \textbf{Dart}: un linguaggio di programmazione orientato agli oggetti sviluppato da Google noto per la sua versabilità. Offre numerosi vantaggi come:
        \begin{itemize}
            \item lo sviluppo di un applicativo \textit{cross-platform} che consente di creare applicazioni native per diversi sistemi da un'unica codebase.
            \item compilazione \textit{Ahead-Of-Time (AOT)} che accelerano lo sviluppo.
            \item supporto alla programmazione asincrona che riduce la gestione dei processi in background.
        \end{itemize}
    \item \textbf{Flutter}: un \textit{framework} basato su Dart per lo sviluppo di applicazioni multi-piattaforma, che offre vantaggi come:
        \begin{itemize}
            \item \textbf{Hot-reload}: che ci permette di visualizzare immediatamente le modifiche al codice durante lo sviluppo senza dover aspettare una nuova compilazione.
            \item \textbf{Compatibilità con material}: ci mette a disposizione un ricco arsenale di widget per creare interfacce utente moderne e accattivanti senza dover scrivere molte righe di codice.
        \end{itemize}
    \item \textbf{Firebase}: un database sviluppato da google e ben integrato con flutter che offre una serie di funzionalità di \textit{backend} pronte all'uso per la propria applicazione.
        \begin{itemize}
            \item \textbf{Authentication}: servizio che offre i servizi di backend e librerie pronte all'uso per autenticare gli utenti nell'applicazione in molteplici modi.
            \item \textbf{Firestore}: un database \textit{NoSQL} orientato ai documenti, che permette di archiviare i documenti in raccolte, che fungono da contenitori
            per organizzare i dati e facilitare le interrogazioni.
        \end{itemize}
\end{itemize}

\subsection{Smart working e strumenti di comunicazione}
\section{Spirito di innovazione aziendale}