\chapter{Realizzazione dell'applicazione}
\label{cap:Realizzazione dell'applicazione}
Il presente capitolo introduce la realizzazione dell'applicazione \textit{Key Wallet App} descrivendo una fase di pianificazione e di analisi per poi proseguire verso i problemi riscontrati durante lo sviluppo, la loro mitigazione e le scelte che sono state intraprese e si conclude con un resoconto su ciò che è stato realizzato.
\section{Pianificazione del lavoro}
Come introdotto nel capitolo 2 l’obiettivo del mio progetto era una valutazione di fattibilità e di potenziale impiego nuove tecnologie per l’azienda in nuovi prodotti. Pertanto mi serviva trovare un compromesso che mi offrisse nella fase di studio rigidità e nella fase di sviluppo e implementazione flessibilità. Quindi per la realizzazione del progetto ho scelto un approccio ibrido tra un modello sequenziale a cascata durante lo studio e la pianificazione, mentre nella realizzazione dell’applicazione ho scelto un modello agile.

\begin{table}[ht]
    \centering
    \begin{tabular}{c|c|c}
          & Settimane  \\
        \hline
        a & b \\
    \end{tabular}
    \caption{Tab}
\end{table}

Durante le prime settimane di stage, è stata applicata una metodologia a cascata, essenziale per stabilire le fondamenta del progetto. Durante la prima settimana ho lavorato a stretto contatto con il mio responsabile per definire gli obiettivi e la struttura del progetto. Dopodiché mi sono concentrato sullo studio del linguaggio Dart e del framework Flutter per poi spostarmi sulla libreria di pointycastle e gli algoritmi di crittografia durante la prime due settimane di stage. Dalla terza settimana fino alla 4 mi sono concentrato sullo sviluppo di piccoli prototipi (es. login finto, generazione di chiavi alla pressione di un tasto e lettura di tag nfc). Questi prototipi sono stati molto utili per effettuare piccoli test sulle funzioni nelle settimane successive e a trovare falle sul progetto immediatamente una tra tutte la lettura di tag nfc. Difatti questi ultimi non possiedono un identificativo fisso sulle carte d’identità elettroniche ma cambia ogni secondo.
\newline
\newline
Una volta completata la fase di prototipazione, il progetto è transitato verso una metodologia agile, in particolare durante la realizzazione dell’applicazione finale. Lo sviluppo agile offre una gestione flessibile dei problemi tecnici, e la possibilità di concentrarsi in modo progressivo sull’implementazione dei singoli componenti del sistema, permettendomi di integrare funzionalità ad ogni iterazione. Questo significava che per le ultime fasi del progetto pianificavo, implementavo e testavo le funzionalità implementate. Permettendomi di reagire immediatamente ai problemi ritrovati, uno tra questi, l’uso di un algoritmo RSA senza padding.
\newline
\newline
L’adozione quindi dell’approccio ibrido ha garantito la copertura formativa e analitica iniziale, consentendo al contempo la gestione flessibile e sicura delle complessità tecniche emerse durante la loro implementazione.


\section{Analisi dei requisiti}
(In questa sezione parlo della fase di analisi per comprendere le tecnologie da utilizzare, come strutturare l'applicazione, i mockup e lo studio delle tecnologie)

\subsection{Requisiti funzionali}
(discuto dei requisiti che riguardano l'utilizzabilità del prodotto finale)

\subsection{Requisiti di qualità}
(discuto degli strumenti e la documentazione da fornire)

\subsection{Requisiti di vincolo}
(discuto delle tecnologie da utilizzare per lo sviluppo dell'applicativo)

\section{Implementazione}

\subsection{Il linguaggio Dart e il framework Flutter}
(In questa sezione illustro i vantaggi e le peculiarità del linguaggio di programmazione Dart e delle scelte architetturali intraprese per lo sviluppo dell'applicazione)

\subsubsection{Il design pattern provider}
(descrivo i vantaggi dell'utilizzo di questo design pattern)

\subsubsection{Architettura a tre livelli}
(descrivo la composizione di questa architettura)

\subsection{la tecnologia NFC}
(in questa sezione discuto di uno degli obiettivi aziendali imposti, ovvero la necessità di associare le chiavi crittografiche ai documenti d'identità)

\subsubsection{Lettura dati nei chip RFID}
(In questa sezione spiego come ho letto determinati dati dai chip RFID e parlo dei problemi riscontrati nel recuperare un dato univoco dalle carte d'identità, in quanto protette da determinati protocolli)

\subsubsection{Mitigazione del problema}
(In questa sezione spiego di come, discutendo con gli stakeholders, siamo riusciti a trovare un compromesso per far si che le chiavi venissero associate ai documenti)

\subsection{Studio degli algoritmi di crittografia}
(In questa sezione descrivo gli algoritmi di crittografia analizzati e scelti per la realizzazione dell'applicativo, in particolar modo dei vantaggi e svantaggi di ognuno)

\subsubsection{Algoritmi a chiave simmetrica}
\subsubsection{Algoritmi a chiave asimmetrica}

\subsection{Firebase} (In questa sezione descrivo il database e la struttura dei dati del progetto)

\subsection{Gestione sicura delle chiavi pubbliche e private}

(In questa sezione descrivo uno dei punti critici del mio progetto ovvero come recuperare la chiave privata che non viene trasmessa online e il suo trasferimento su un nuovo dispositivo in caso di smarrimento)

\subsection{Realizzazione di una chat \textit{end-to-end}}
(In questa sezione discuto di come ho realizzato una chat utente per dimostrare come i messaggi inseriti nell'applicativo vengano criptati e decifrati per rispettare i requisiti, e i problemi riscontrati)

\subsubsection{L'algoritmo RSA}
\subsubsection{Problemi di RSA}
\subsubsection{Mitigazione del problema tramite RSAOAEP}

\section{Risultati Raggiunti}
(in questa sezione parlo dei risultati e raggiunti, delle funzioni implementate nell'applicazione come ad esempio la ricerca, la conversazione, e come le chiavi sono state associate ai documenti)

\subsection{Risultati quantitativi}
\subsection{Risultati qualitativi}