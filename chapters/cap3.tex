\chapter{Realizzazione dell'applicazione}
\label{cap:Realizzazione dell'applicazione}
Il presente capitolo introduce la realizzazione dell'applicazione \textit{Key Wallet App} descrivendo una fase di pianificazione e di analisi per poi proseguire verso i problemi riscontrati durante lo sviluppo, la loro mitigazione e le scelte che sono state intraprese e si conclude con un resoconto su ciò che è stato realizzato.
\section{Pianificazione del lavoro}
Come introdotto nel capitolo 2 l’obiettivo del mio progetto era una valutazione di fattibilità e di potenziale impiego nuove tecnologie per l’azienda in nuovi prodotti. Pertanto mi serviva trovare un compromesso che mi offrisse nella fase di studio rigidità e nella fase di sviluppo e implementazione flessibilità. Quindi per la realizzazione del progetto ho scelto un approccio ibrido tra un modello sequenziale a cascata durante lo studio e la pianificazione, mentre nella realizzazione dell’applicazione ho scelto un modello agile.

\begin{table}[ht]
    \centering
    \begin{tabular}{|p{6cm}|c|c|c|c|c|c|c|c|c|}
        \hline
        \textbf{Attività} & \multicolumn{8}{|c|}{\textbf{Settimane}} & \textbf{Ore} \\
        \hline
         & 1 & 2 & 3 & 4 & 5 & 6 & 7 & 8 & \\
        \hline
        Ripasso costrutti di Java & X & & & & & & & & 5 \\
        \hline
        Studio di Dart & X & X & & & & & & & 30 \\
        \hline
        Studio di Flutter & & X & X & & & & & & 40 \\
        \hline
        Studio algoritmi di criptazione &  & X & & X & & X & & & 30 \\
        \hline
        Analisi del problema & X &  & &  & & & & & 10 \\
        \hline
        Progettazione della piattaforma &  &  &  & X & & & & & 25 \\
        \hline
        Sviluppo maschera di \textit{login} &  &  & X & & & & & & 5 \\
        \hline
        Sviluppo di un prototipo che genera chiavi &  &  & X & & & & & & 30 \\
        \hline
        Sviluppo applicazione finale &  &  & & & X & X & X & & 100 \\
        \hline
        Stesura finale della specifica tecnica &  &  & & &  & & X & X & 20 \\
        \hline
        \textit{Live demo} e presentazione finale &  &  & & & & & & X & 5 \\
        \hline
        \textbf{totale ore} & \multicolumn{9}{|c|}{300}\\
        \hline
    \end{tabular}
    \caption{Pianificazione del lavoro durante le 8 settimane}
\end{table}

Durante le prime settimane di stage, è stata applicata una metodologia a cascata, essenziale per stabilire le fondamenta del progetto. Durante la prima settimana ho lavorato a stretto contatto con il mio responsabile per definire gli obiettivi e la struttura del progetto. Dopodiché mi sono concentrato sullo studio del linguaggio Dart e del framework Flutter per poi spostarmi sulla libreria di pointycastle e gli algoritmi di crittografia durante la prime due settimane di stage. Dalla terza settimana fino alla 4 mi sono concentrato sullo sviluppo di piccoli prototipi (es. login finto, generazione di chiavi alla pressione di un tasto e lettura di tag nfc). Questi prototipi sono stati molto utili per effettuare piccoli test sulle funzioni nelle settimane successive e a trovare falle sul progetto immediatamente una tra tutte la lettura di tag nfc. Difatti questi ultimi non possiedono un identificativo fisso sulle carte d’identità elettroniche ma cambia ogni secondo.
\newline
\newline
Una volta completata la fase di prototipazione, il progetto è transitato verso una metodologia agile, in particolare durante la realizzazione dell’applicazione finale. Lo sviluppo agile offre una gestione flessibile dei problemi tecnici, e la possibilità di concentrarsi in modo progressivo sull’implementazione dei singoli componenti del sistema, permettendomi di integrare funzionalità ad ogni iterazione. Questo significava che per le ultime fasi del progetto pianificavo, implementavo e testavo le funzionalità implementate. Permettendomi di reagire immediatamente ai problemi ritrovati, uno tra questi, l’uso di un algoritmo RSA senza padding.
\newline
\newline
L’adozione quindi dell’approccio ibrido ha garantito la copertura formativa e analitica iniziale, consentendo al contempo la gestione flessibile e sicura delle complessità tecniche emerse durante la loro implementazione.


\section{Analisi dei requisiti}
La fase di analisi svolta durante le prime settimane del percorso di \textit{stage}, sono state cruciali per definire l’architettura applicativa e le tecnologie da utilizzare. Gli incontri con il \textit{tutor} hanno rappresentato un fulcro importante per lo sviluppo e la definizione del progetto. Le diverse iterazioni hanno permesso di delineare le linee guida e velocizzare le scelte tecnologiche, ad esempio l’uso di librerie specifiche come \texttt{pointycastle} per la crittografia o \texttt{flutter\_secure\_storage} per la gestione locale della chiave privata. A seguito di queste indicazione la fase di analisi è stata focalizzata sullo studio di fattibilità delle tecnologie e sullo sviluppo di piccoli prototipi. 

\newpage

\subsection{Requisiti funzionali}
I requisiti funzionali definiscono le funzionalità e i servizi specifici che l’applicazione deve fornire per conseguire gli obiettivi del progetto.
Ogni requisito funzionale è codificato in questo modo: \textbf{RF[Numero]}.

\begin{table}[ht]
    \centering
    \begin{tabular}{|c|p{10cm}|}
        \hline
        \textbf{Codice} & \textbf{Descrizione} \\
        \hline
        \textbf{RF1} & Generazione di una coppia di chiavi pubbliche e private\\
        \hline
        \textbf{RF2} & Generazione ed eliminazione di \textit{wallet} contenenti le coppie di chiavi\\
        \hline
        \textbf{RF3} & Implementazione della parte di accesso e registrazione tramite appoggio di \textit{database}\\
        \hline
        \textbf{RF4} & Implementazione di un meccanismo di storage sicuro che permetta agli utenti di salvare le chiavi private sul \textit{Keystore/Keychain} del dispositivo\\
        \hline
        \textbf{RF5} & Riuscire a criptare un messaggio inserito nell'applicazione tramite la chiave pubblica di un altro utente\\
        \hline
        \textbf{RF6} & Riuscire a decifrare un messaggio tramite la propria chiave privata personale\\
        \hline
        \textbf{RF7} & Implementazione di un meccanismo di recupero \textit{wallet} \\
        \hline
        \textbf{RF8} & Implementazione di un meccanismo di lettura di \textit{tag NFC} per associare un documento a un \textit{wallet}\\
        \hline
    \end{tabular}
    \caption{Requisiti funzionali}
\end{table}

\subsection{Requisiti di qualità}
I requisiti di qualità stabiliscono degli \textit{standard} secondo cui tali funzionalità del prodotto devono essere realizzate ed eseguite, assicurando un elevato livello di affidabilità e coerenza del sistema.  Quest’ultimi sono stati definiti all’inizio del percorso di \textit{stage} e sono stati misurati alla fine come criteri per la valutazione del progetto.
Ogni requisito di qualità è codificato in questo modo: \textbf{RQ[Numero]}.

\begin{table}[ht]
    \centering
    \begin{tabular}{|c|p{10cm}|}
        \hline
        \textbf{Codice} & \textbf{Descrizione} \\
        \hline
        \textbf{RQ1} & Assicurare un'accuratezza nella generazione e crittografia delle chiavi superiore al 95\% \\
        \hline
        \textbf{RQ2} & Mantenere il tempo di risposta per la generazione del \textit{wallet} al di sotto dei 2 secondi\\
        \hline
        \textbf{RQ3} & Raggiungere una copertura di test automatici del 70\% per le principali funzionalità\\
        \hline
    \end{tabular}
    \caption{Requisiti di qualità}
\end{table}

Ognuno di questi requisiti è stato verificato attraverso lo sviluppo di test automatici.


\subsection{Requisiti di vincolo}
I requisiti di vincolo definiscono le limitazioni operative e tecnologiche cui il progetto ha dovuto attenersi, influenzando direttamente le scelte architetturali e l'ambito di sviluppo. 
Abbiamo già discusso dei vincoli del progetto nella sezione 2.4.2 Pertanto mi limito ad elencarli in questa sezione.
Ogni requisito di vincolo è codificato in questo modo: \textbf{RV[Numero]}.

\begin{table}[ht]
    \centering
    \begin{tabular}{|c|p{10cm}|}
        \hline
        \textbf{Codice} & \textbf{Descrizione} \\
        \hline
        \textbf{RV1} & L'applicazione finale deve essere sviluppata tramite il \textit{framework} Flutter e il linguaggio Dart\\
        \hline
        \textbf{RV2} & L'applicazione finale deve appoggiarsi a Firebase come appoggio per una base di dati \\
        \hline
        \textbf{RV3} & Le chiavi private generate non devono essere trasmesse \textit{online} ma devono essere cutodite in spazi sicuri del dispositivo come il \textit{Keychain} e \textit{Keystore}\\
        \hline
        \textbf{RV4} & Ogni documento deve essere associato a unico wallet\\
        \hline
    \end{tabular}
    \caption{Requisiti di vincolo}
\end{table}


\section{Implementazione}

\subsection{Il linguaggio Dart e il framework Flutter}
(In questa sezione illustro i vantaggi e le peculiarità del linguaggio di programmazione Dart e delle scelte architetturali intraprese per lo sviluppo dell'applicazione)

\subsubsection{Il design pattern provider}
(descrivo i vantaggi dell'utilizzo di questo design pattern)

\subsubsection{Architettura a tre livelli}
(descrivo la composizione di questa architettura)

\subsection{la tecnologia NFC}
(in questa sezione discuto di uno degli obiettivi aziendali imposti, ovvero la necessità di associare le chiavi crittografiche ai documenti d'identità)

\subsubsection{Lettura dati nei chip RFID}
(In questa sezione spiego come ho letto determinati dati dai chip RFID e parlo dei problemi riscontrati nel recuperare un dato univoco dalle carte d'identità, in quanto protette da determinati protocolli)

\subsubsection{Mitigazione del problema}
(In questa sezione spiego di come, discutendo con gli stakeholders, siamo riusciti a trovare un compromesso per far si che le chiavi venissero associate ai documenti)

\subsection{Studio degli algoritmi di crittografia}
(In questa sezione descrivo gli algoritmi di crittografia analizzati e scelti per la realizzazione dell'applicativo, in particolar modo dei vantaggi e svantaggi di ognuno)

\subsubsection{Algoritmi a chiave simmetrica}
\subsubsection{Algoritmi a chiave asimmetrica}

\subsection{Firebase} (In questa sezione descrivo il database e la struttura dei dati del progetto)

\subsection{Gestione sicura delle chiavi pubbliche e private}

(In questa sezione descrivo uno dei punti critici del mio progetto ovvero come recuperare la chiave privata che non viene trasmessa online e il suo trasferimento su un nuovo dispositivo in caso di smarrimento)

\subsection{Realizzazione di una chat \textit{end-to-end}}
(In questa sezione discuto di come ho realizzato una chat utente per dimostrare come i messaggi inseriti nell'applicativo vengano criptati e decifrati per rispettare i requisiti, e i problemi riscontrati)

\subsubsection{L'algoritmo RSA}
\subsubsection{Problemi di RSA}
\subsubsection{Mitigazione del problema tramite RSAOAEP}

\section{Risultati Raggiunti}
(in questa sezione parlo dei risultati e raggiunti, delle funzioni implementate nell'applicazione come ad esempio la ricerca, la conversazione, e come le chiavi sono state associate ai documenti)

\subsection{Risultati quantitativi}
\subsection{Risultati qualitativi}